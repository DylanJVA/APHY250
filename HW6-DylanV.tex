\documentclass{article}
\usepackage{amsmath, amssymb, amsthm}
\usepackage{inputenc}
\usepackage{geometry}
\usepackage{graphicx}
\geometry{legalpaper, portrait, margin=1in}
\graphicspath{{images/}}
\begin{document}
APHY250

Homework 6 

Dylan VanAllen 

Chapter 8. Two-Coordinate Vibrations
\begin{enumerate}
    \item (8.1) For the symmetric system (fig 8.1) with weak coupling \(S << s\) show that the fraction by which the larger mode frequency exceeds the smaller mode frequency is approximately \(\frac{S}{s}\).
    \begin{figure}[h!]
        \centering
        \includegraphics[width=12cm]{8.1.jpg}
    \end{figure}
    \begin{align*}
        \omega_1 &\equiv (s/m)^{1/2} \tag{8.3} \\
        \omega_2 &\equiv [(s+2S)/m]^{1/2} \tag{8.4} \\
        \frac{\omega_2}{\omega_1} &=\frac{[(s+2S)/m]^{1/2}}{(s/m)^{1/2}} \\
        &= \left(\frac{(s+2S)/m}{(s/m)}\right)^{1/2} \\
        &= \left(\frac{(s+2S)}{s}\right)^{1/2} \\
        &= \left(1+\frac{2S}{s}\right)^{1/2} \\
        &=\frac{S}{s}+\frac{1}{2}
    \end{align*}
    \item (8.2) At a certain instant during the vibration of the symmetric system (fig 8.1) the mode coordinates have the values \(q_1 = 1.36\times10^{-3} \ \text{kg}^{1/2}\)m and \(q_2 = -0.34\times10^{-3} \ \text{kg}^{1/2}\)m. Calculate \(\psi_1\) and \(\psi_2\) at the same instant if \(m=0.020 \ kg\). 
    \begin{align*}
        
    \end{align*}
    \item (8.4) If the system in fig 8.8 has \(m_1 = 3m_2\) and \(s_1 = s_2=2S\), find:
    \begin{figure}[h!]
        \centering
        \includegraphics[width=12cm]{8.8.jpg}
    \end{figure}
    \begin{itemize}
        \item (a) The amplitude ratio \(\frac{A_2}{A_1}\) for the 'in-phase' mode
        \item (b) the amplitude ratio \(\frac{A_2}{A_1}\) for the 'in-antiphase' mode
        \item (c) the ratio of the mode frequencies
    \end{itemize}
    \item (8.6) Two identical simple pendulums are connected by a light spring attached to the bobs. Each bob has a mass of \(1.00\) kg, and the stiffness of the coupling spring is \(0.800\) N\(\text{m}^-1\). When one pendulum is clamped, the period of the other is found to be \(1.25\) s. Find the periods of the two modes of the system with the clamp removed. 
    \item Consider 4 identical simple pendulums, equally spaced on a row, and all masses are linked by identical light springswith a force constant \(s\) such that when not in vibrations all pendulums remain vertical. Use the method of transverse standing waves, find the four normal modes of this four-coordinate system and comment on their angular frequencies.

\end{enumerate}
\end{document}